\chapter{Gesetze und Richtlinien}
Die in dieser Arbeit beschriebene Audit-Log-Implementierung kann in verschiedenen Bereichen mit persönlichen Informationen angewandt werden. Insbesondere Daten im Medizin- und Finanzbereich unterliegen strengen rechtlichen Auflagen.\\
Da digitale Dokumente leicht manipuliert werden können, gibt es Gesetzte und Richtlinien, die die Prüfbarkeit der Echtheit digitaler Dokumente erfordern. Zusätzlich müssen Zugriffe beschränkt und dokumentiert werden. Gesetze und Richtlinien, die im Bezug auf das Audit-Log angewandt werden können, werden in diesem Kapitel erläutert. \cite{1966944}

\section{Nationales Recht}
Das Handelsgesetzbuch schreibt den Nachweis über Handelsbücher, Inventare, Eröffnungsbilanzen, Jahresabschlüsse, Einzelabschlüsse, Handelsbriefe und Belege für Buchungen vor. Die Nachweispflicht erlischt nach sechs oder zehn Jahren, je nach Dokumententyp.\cite{11}\\
Das Umsatzsteuergesetz schreibt das Aufbewahren von ausgestellten Rechnungen für mindestens zwei Jahre vor.\cite{12}\\
Die Abgabenordnung schreibt die Aufbewahrung aller steuerrelevanten Dokumente und deren Nachweis vor. Die Nachweispflicht erlöscht nach sechs oder zehn Jahren, je nach Dokumententyp.\cite{13}\\
Die Grundsätze zur ordnungsmäßigen Führung und Aufbewahrung von Büchern, Aufzeichnungen und Unterlagen in elektronischer Form, sowie zum Datenzugriff (\acs{GoBD}), ist eine Ergänzung zur Abgabenordnung und schreibt ebenfalls die Aufbewahrung steuerrelevanter Dokumente vor. Zusätzlich werden  Grundbucheinträge und allgemeine Geschäftsvorfälle eingeschlossen. Diese, 2014 erlassene, Verordnung ist im Vergleich mit der Abgabenordnung recht aktuell und geht explizit auch auf elektronische Datenverwaltung ein. Es werden Anforderungen an den Datenzugriff und die dauerhafte Verfügbarkeit gestellt. Die Anforderungen an die \acs{EDV} sind aber trotzdem sehr vage formuliert. Beispielsweise schreibt § 3.1 des \acs{GoBD} "`[...]angewandte Buchführungs- oder Aufzeichnungsverfahren müssen nachvollziehbar sein."'\cite{141} kein explizites Verfahren vor.\cite{14}

\section{Europa Recht}
Die europäische Datenschutz-Grundverordnung (\acs{DSGVO}) schreibt sowohl die Aufbewahrung bestimmter Dokumente, als auch deren Löschung vor. Die Aufbewahrung personenbezogener Daten wird mit sogenannten "`Löschfristen"' limitiert. Oft sind die Löschfristen mit der verpflichteten Aufbewahrungszeit gleichzusetzen.\cite{15}\\

\section{Richtlinien}

Der \acs{EU-GMP}(European Good Manufacturing Practice) Leitfaden schreibt zu computergestützten Systemen die Erstellung und die Auditierung von Audit-Logs expliziert vor, formuliert aber keine technischen Vorgaben.\cite{20} Die Richtlinie wurde in Deutschland 2003 in die Arzneimittel- und Wirkstoffherstellungsverordnung des  Bundesministerium für Gesundheit mit aufgenommen. \cite{19}\\

Die Bundesanstalt für Finanzdienstleistungsaufsicht (\acs{BaFin}) veröffentlichte im September 2018 die Kapitalverwaltungsaufsichtliche Anforderungen an die IT (\acs{KAIT}). Analog zur \acs{KAIT} veröffentlichte die \acs{BaFin} im September 2018 die Bankaufsichtliche Anforderungen an die IT (\acs{BAIT}) und im Oktober 2018 die Versicherungsaufsichtliche Anforderungen an die IT (\acs{VAIT}).\\
In diesen wird speziell auf den Umgang und die Verwaltung von Daten in den jeweiligen Unternehmensbereichen eingegangen, ohne jedoch konkrete Vorschläge von Verfahren zu definieren. Oft werden Anforderungen mit 
"`[\ldots] den Stand der Technik berücksichtigende Informationssicherheitsrichtlinien und Informationssicherheitsprozesse [\ldots]"' \cite{161} umschrieben. Es werden jedoch kryptographische Maßnahmen zum Schutz und zur Verifikation von Daten in \acs{IT} Systemen verlangt. Zudem wird eine Authentisierung und Protokollierung gefordert.\cite{16}\cite{17}\cite{18}

\subsection{TR-ESOR}
Das Bundesamt für Sicherheit in der Informationstechnik (\acs{BSI}) veröffentlichte die Publikation "`\acs{BSI} TR-03125 Beweiswerterhaltung kryptographisch signierter Dokumente"' kurz \acs{TR-ESOR}. Diese soll als Leitfaden gesehen werden und gibt präzise Vorschläge für Verschlüsselungs- und Signierung-Verfahren für Dokumente vor. Auch für die Architektur der Anwendung werden Vorgaben formuliert. Datenübertragung, Schnittstellen und Protokolle sind ausformuliert. Diese Richtlinie befindet sich zur Zeit dieser Arbeit noch in Entwicklung. \\
\acs{TR-ESOR} schlägt eine Dreiteilung der Sicherheitsstruktur vor. Die drei Teile bestehen aus einem \acs{API}-Modul, welches die Anwendung von den \acs{ECM}/Langzeitspeichersystemen abstrahiert. Ein Kryptomodul welches Funktionen für die Erstellung und Prüfung kryptographischer Sicherungsmittel bereitstellt. Zuletzt ein Signierungsmodul, welches als Bindeglied zwischen dem Kryptomodul, dem \acs{API}-Modul und dem \acs{ECM} dient. Hier werden die Kryptographischen Funktionen auf die zu verwaltenden Daten angewandt und entsprechend der \acs{API} weitergeleitet und verarbeitet.\\
Der Anwendungsbereich von \acs{TR-ESOR} beschreibt genau die Software die Gegenstand der Thesis ist. Die in dieser Arbeit vorgelegte Audit-Log-Implementierung hat sich an die Vorgaben von \acs{TR-ESOR} gehalten. Einzelne Vorgaben die den Rahmen dieser Thesis überschritten hätten wurden so behandelt, dass entsprechende Mechanismen später problemlos integriert werden können.\\
\acs{TR-ESOR} schlägt die Nutzung qualifizierter elektronischer Signaturen (\acs{QES}) vor. Die \acs{QES} kann von europäisch lizenzierten Verifikationsquellen erstellt werden und gilt somit als vertrauenswürdig. Die Benutzung solcher Signaturen ist nicht Gegenstand dieser Thesis, jedoch ist deren Nutzung und Einsatz für die Zukunft geplant. Gleiches gilt für die Nutzung von e-Cards. e-Cards sind physikalische und verifizierte Karten mit denen vertrauenswürdige benutzerauthentisierte Signaturen erstellt werden können. \cite{2}
