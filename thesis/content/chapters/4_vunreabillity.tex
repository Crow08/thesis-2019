\chapter{Vulnerabilität}
In größeren Softwaresystemen ist das Auditsystem nur eine Teilkomponente des Geasmtsystems. Im Betrieb wird über Schnittstellen mit zum Teil externen Datenquellen und anderen Akteuren kommuniziert. Aber nicht alle diese Akteure sind uneingeschränkt vertrauenswürdig. Um dennoch mit diesen arbeiten zu können, benötigt man Kontrollmechanismen, um externe Daten validieren zu können.\cite{8494085}\\
Eine Bedrohung bzw. Verwundbarkeit der Softwarekomponente kann entweder von außen, durch öffentliche Schnittstellen, oder von innen, mit Zugriff auf Systeminformationen wie  private Schlüssel, erfolgen.\cite{8574150}

\section{Angriffe von Außen}
Das verwendete Speichermedium ist ein großer Risikofaktor, da dieses Daten unabhängig verwaltet. Es muss sichergestellt sein, das abgelegte Daten unverändert wieder abgerufen werden können. Wenn ein Angreifer Zugriff auf das externe Speichermedium erlangt, könnte dieser Daten nach Belieben manipulieren. Auch unbeabsichtigte Manipulation durch Datenkorruption ist eine Gefahr. Das Schützen des Speichermediums vor unbefugten Manipulationen ist schwierig und nicht Teil dieser Arbeit. Es können aber Verfahren eingesetzt werden, die nach dem Abruf von Daten, deren unveränderten Zustand validieren kann. Die Validierung kann mithilfe digitaler Signaturen oder anderer asynchroner kryptographischer Verfahren garantiert werden. Digitale Signaturen können von vertrauenswürdigen externen Quellen generiert und validiert werden und somit den Inhalt der Daten validieren.\\
Der zweite Risikofaktor ist jeder reguläre Akteur oder Nutzer des Systems. Aktionen, die durch offizielle Schnittstellen zur Verfügung gestellt werden, müssen mithilfe von Nutzungsbeschränkungen geschützt werden. Alle Akteure, die ein Audit-Event auslösen, müssen sich authentisieren. Authentisierung ist der eindeutige Nachweis, dass es sich um einen bekannten Akteur handelt. Zudem muss der Akteur autorisiert sein. Autorisierung beschreibt das Einräumen von speziellen Rechten zur Interaktion mit dem System. Nur ein bekannter und berechtigter Akteur darf ein Audit-Event auslösen.\\
Zusammenfassend kann man folgern, dass jede Interaktion mit einem nicht systemeigenen Akteur, nicht vertrauenswürdig ist und für jede bereitgestellte Funktionalität müssen entsprechende Kontrollmechanismen vor Missbrauch schützen. Auch jede vom System genutzte externe Ressource muss auf Integrität geprüft werden.\cite{8494085}

\section{Angriffe von Innen}
Selbst wenn Daten mittels digitaler Signaturen validiert sind, ist es immer noch möglich Daten zu fälschen bzw. zu manipulieren, wenn man Zugriff auf den Signierungsschlüssel hat. Im Fall des Audit-Logs bedeutet dies beispielsweise, dass ein Systemadministrator die Möglichkeit hat, unbemerkt Daten zu manipulieren. Das Verhindern von Manipulationen durch berechtigte Benutzer mit vollem Zugriff auf das System ist äußerst schwierig. Es gibt jedoch Mechanismen die Daten zu schützen, auch wenn ein Nutzer die Berechtigung hat Änderungen vorzunehmen. Ein solcher Mechanismus ist es eine Hashstruktur zu bilden, bei der jedes neue Element auf die Daten seiner Vorgänger verweist und somit jede Änderung an bestehenden Daten sofort zu einer Inkonsistenz in der Hashberechnung führt, die nachgewiesen werden kann. Je mehr Daten für die Hashberechnung einbezogen werden, desto sicherer wird es deren Integrität zu beweisen. Aber auch diese Maßnahme ist nicht unfehlbar, denn der Angreifer hat die Möglichkeit die gesamte Hashverifikation mit seinen manipulierten Daten neu zu berechnen. Dies bedeutet aber einen enormen Aufwand und erfordert beträchtliche Zeit und Rechen-Ressourcen und kann so wiederum bemerkt werden. Das System kann durch die inkrementelle Abhängigkeit der Daten keine neuen Daten verarbeiten bevor die Neuberechnung nicht abgeschlossen ist. \cite{8494085} \cite{8574150} \cite{3286985}
