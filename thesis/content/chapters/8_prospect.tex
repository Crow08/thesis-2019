\chapter{Ausblick}
Audit-Logs und die Verifizierbarkeit von Daten sind in der Medizin- und Finanzbranche bereits fester Bestandteil von Business Process Management Software. Auch in anderen Bereichen ist deren Einsatz sinnvoll und wird vermehrt verwendet.
Das im Rahmen dieser Thesis entwickelte Audit-Log-Modul erfüllt die gestellten Anforderungen und kann in neverpile eureka produktiv eingesetzt werden. Alle Funktionen sind mit nach aktuellen Gesetzen und Richtlinien konformen Lösungen implementiert und können direkt verwendet werden.
Anpassungen an fachliche oder kundenspezifische Anforderungen sind in der Architektur vorgesehen und können bei Bedarf ergänzt werden.\\
Im Speziellen kann die Absicherung der Datenstruktur durch einen "`Proof of Work"' Algorithmus ersetzt werden. Dieser bietet einen größeren Schutz vor Manipulationen, erfordert aber erhöhten Rechenaufwand. Auch die Einbindung qualifizierter digitaler Signaturen wäre eine denkbare Erweiterung. Diese erhöhen die Verlässlichkeit der digitalen Signaturen.\\
Optimierungen hinsichtlich Datenspeicherung und Blockbildung können ebenfalls an kundenspezifische Installationen angepasst werden. Dabei kommt es auf das Zusammenspiel eingehender Events pro Minute und dem verwendeten Speichersystem an. Die Blockbildung muss, bei hoher Last, mehr Events in Blöcke zusammenfassen, um das Speichersystem nicht zu überlasten. Auch muss darauf geachtet werden, dass die Blöcke nicht zu groß oder zu klein werden, um Zugriffszeiten sowohl bei Einzelzugriffen als auch bei kompletten Verifikationen gering zu halten.\\
Durch die Modularität des Systems kann jedes der verwendeten kryptographischen Verfahren bei Auftreten einer Sicherheitslücke, problemlos durch ein anderes ersetzt werden. Somit bleibt das Log-System zukunftssicher und flexibel.
