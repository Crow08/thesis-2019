\chapter{Fazit}
Der Fokus der Thesis hat sich im Laufe der Analyse weg vom Blockchain-Ansatz, hin zu Hashchain-Technologien verschoben. Das Ziel der Thesis, ein durch Kryptographie abgesicherter Audit-Log zu entwerfen und zu implementieren, wurde dadurch nicht beeinflusst. Zu Beginn wurde Blockchain als technologische Grundlage gewählt. Die Analyse hat gezeigt, dass die inhärenten Vor- und Nachteil dieses Ansatzes nicht gänzlich auf die Anforderungen des Zielsystems neverpile eureka passen.\\
Da das Audit-Log-Facet nicht das Herzstück der Gesamtsoftware darstellt, diese aber um zusätzliche Funktionalität erweitert, sollte das Audit-Log so einfach wie möglich gehalten werden. Die vom Blockchain-Ansatz verwendeten "`Proof Of Work"'-Algorithmen, führen zu enormen Overhead und beeinflussen die Performance des Gesamtsystems negativ. Das simultane und somit redundante Arbeiten aller Nodes am selben Arbeitspaket erzeugt ebenso einen Overhead, der mit der in dieser Arbeit beschriebenen Lösung vermieden wurde. Die nötige Kommunikation zwischen einzelnen Instanzen ist, im Vergleich mit Blockchain, ebenfalls deutlich reduziert. Mit der Nutzung eines gemeinsamen Object Stores entfällt außerdem das Ablegen einer lokalen Kopie der Datenstruktur in jeder Instanz. Die Verwendung digitaler Signaturen als Ankerpunkt wird nur in der vorliegenden Lösung benötigt um Sicherheit zu gewärleisten. Bei Blockchain ist durch die starke Absicherung mittels Konsensalgorithmen keine weitere Absicherung notwendig.\\
Trotz dieser Unterschiede schützen beide Ansätze die Daten durch die inhärente Abhängigkeit in der verwendeten Datenstruktur. Dies führt in beiden Fällen dazu, dass jeder neue Eintrag in der Datenstruktur die Sicherheit der bereits vorhandenen Einträge erhöht.\\
