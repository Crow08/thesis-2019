\chapter*{Abstract\markboth{Abstract}{}}
\addcontentsline{toc}{chapter}{Abstract}

The goal of this bachelor thesis is the use of Hashchain technologies to secure audit logs. The resulting artifact is a software module that creates and manages audit logs for sensitive data. The module is embedded in neverpile eureka, a long-term archive system developed by levigo solutions gmbh. The audit log should be protected against attacks and manipulation via cryptographic methods. It also enables the validation of the audited data. The implementation uses a combination of different cryptographic methods and data structures, such as Merkle trees and hash chains. The design is based on the blockchain principle. The similarities and differences are shown in this document. In-memory computing ensures, that the module can create a consistent and fault-tolerant audit log in a scalable, distributed system.
\\
Das Ziel der vorliegenden Bachelorarbeit ist die Verwendung von Hashchain-Technologien zur Absicherung von Audit-Logs. Das resultierende Artefakt ist ein Softwaremodul, dass Audit-Logs für sensible Daten erstellt und verwaltet. Das Modul wird in neverpile eureka eingebettet, einem Langzeit-Archiv-System der levigo solutions gmbh. Das Audit-Log soll über kryptographische Verfahren vor Angriffen und Manipulation geschützt werden. Außerdem wird die Validierung der auditierten Daten ermöglicht. Für die Umsetzung wird dazu eine Kombination verschiedener kryptographischer Verfahren und Datenstrukturen, wie Merkle trees und Hashchains verwendet. Der Entwurf orientiert sich dabei am Blockchain-Prinzip. Die Gemeinsamkeiten und Unterschiede werden in dieser Arbeit aufgezeigt. Mittels "`In-Memory Computing"' soll sichergestellt werden, dass das Modul in einem skalierbaren, verteilten System, ein konsistentes und fehlertolerantes Audit-Log erstellen kann.